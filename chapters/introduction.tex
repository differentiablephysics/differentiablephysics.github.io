\documentclass[book.tex]{subfiles}

\begin{document}

\chapter{What is Differentiable Physics?}
\label{ch:intro} 
The heart of physics is the construction of mathematical theories which accurately predict the behavior of real systems. If your background is in machine learning, you might pause and ask, doesn't machine learning also aim to achieve the same result? At a high level, the answer is yes. The difference is that accepted models of physical reality are far more predictive than traditional machine learning. Why is this the case? The reason is invariants are much more tightly defined and controlled for physical models than for machine learned models, which are often designed for ease of optimization rather than for fidelity to physical reality. The goal of this book is to help bridge the gap between accurate causal models of reality and machine learning. Let's start by making a definition of a physical theory.

In a mathematical sense, we can view a physical theory as a mathematical object consisting of the following parts

\begin{enumerate}
    \item States $\mathcal{S}$: This is a mathematical space which contains different possible configurations of the system we are interested in modeling.
    \item Symmetries $\mathcal{G}$: A collection of symmetries which the system must satisfy. These could be properties such as rotational or translational invariances (or equivariances) which posit that certain properties of the physical system are unchanged (or change in a structured fashion) under these transformations. Each such symmetry is associated with an according conservation law for the system by Noether's theorem. For examples, one such conservation law could be that energy is known to be conserved by the system.
    \item Operators $\mathcal{O}$: A collection of functions which map states in $\mathcal{S}$ to other states in $\mathcal{S}$. The most important operator for a physical system is often the time-evolution operator which governs how the system changes over time.
    \item Parameters $\mathcal{W}$: The set of parameters which govern the physical system. These could for example be physical constants such as the gravitational constant $g$. We've chosen the evocative name $\mathcal{W}$ for this parameter set to remind you of weights since these parameters have to be "learned" from physical experiments.
    \item Domain of Applicability $\mathcal{C}$: Physical theories typically have constraints on when they are applied. We model the domain of applicability as a set of constraints $\mathcal{C}$.
    \item Hyperparameters $\mathcal{HP}$: A hyperparameter of a physical theory is parameter that appears in a constraint on the system state or system parameters. You can think of a hyperparameter as a parameter that controls parameters. For example, if mass $m$ is a parameter of a physical model, then there might be a hyperparameters $M_{\textrm{max}}$ which limits $m < M_{\textrm{max}}$
    \item Observables $\mathcal{B}$: An observable is a function that maps from state space $\mathcal{S}$ to some other mathematical space, typically $\mathcal{R}$. Think of an observable as a physical quantity you can measure by observing the evolution of the physical system over time. 
\end{enumerate}

Put together, we can view a physical theory as a data structure, a tuple 
\begin{align}
\textrm{PhysicalTheory} = (\mathcal{S}, \mathcal{G}, \mathcal{O}, \mathcal{W}, \mathcal{C}, \mathcal{HP}, \mathcal{B})
\end{align} 
which models an aspect of reality. An arbitrary physical theory is unlikely to be predictive. Part of the art of physics is constructing physical theories which are predictive of reality as we know it. Over the course of this book, we introduce a variety of techniques for constructing useful physical theories. Some techniques will draw upon classical physics, while others will draw from machine learning and the disciplines of type theory, and differentiable programming. 


Types are a technique originating in computer science and logic which associate to each term in a program a descriptive type. When constructing a physical theory, we will take care to specify the core types that populate that physical theory. For example, here are a few types we will see repeatedly in the book
\begin{align}
    \mathbb{R}^3, \mathbb{R}^{1,3} L^2(\mathbb{R}^3, \mathcal{C}), \mathcal{B}(\mathcal{H})
\end{align}
which respectively represent the types of three dimensional space, Minkowski space, quantum wave functions in three dimensions, and quantum observables. By identifying the core types in a physical system, we can transform the system into a rich implementation in a programming language with a suitably powerful type theory. Note that by our earlier definition, a physical theory is itself a composition type.

The last tool we will use to model physical systems is differentiable programming. The last decade of machine learning research has revealed that essentially arbitrarily sophisticated programs can be differentiated by using the tools of automatic differentiation. As a result, tools like differential equation solvers, and physical integrators can be leveraged as parts of a machine learning model. By leveraging the universal approximation capability of neural networks, which allow neural representations to represent arbitrarily complex functions, and combining with rich typed differentiable programs, we may achieve new insights into physical reality.

%\subsection{Physical Theories and Differentiable Programs}

%If you're a physicist, you might be wondering what the point of having a formalism for physical theories. You already work with plenty of physical theories already without defining them. What's the difference? To discuss this more carefully, we introduce one additional concept.

Formally, a differentiable program is a computer program controlled by a set of \textit{parameters} which can be \textit{differentiated} with respect to any of its parameters. Let's introduce a simple example of a differentiable program. We will write this program in pseudo-code similar to a high-level language like Python.

\begin{lstlisting}
# Parameters
$w_1$: $\mathbb{R}$, $w_2$: $\mathbb{R}$
def f(x: $\mathbb{R}$, y: $\mathbb{R}$) $\to$ $\mathbb{R}$:
  return $w_1$ * x + $w_2$ * y
grad(f, $w_1$)
\end{lstlisting}

We have a function $f$ which accepts two arguments, $x$, and $y$ which are both real numbers. We have two parameters $w_1$ and $w_2$. So far this should look pretty similar to a regular program you've seen before, with the difference that we can have variables that take values in $\mathbb{R}$ (recall that $\mathbb{R}$ denotes the set of real numbers). In programming languages, representing values in $\mathbb{R}$ isn't typically possible since infinite decimal representations can't be stored on a computer, but for mathematical simplicity it's quite convenient to be able to manipulate real numbers.

There is one new operation we can do with differentiable programs that we can't typically do with regular programs: take derivatives. That is, the following operation is valid in a differentiable program.

\begin{lstlisting}
grad = $\nabla$(f, $w_1$)
\end{lstlisting}

In familiar mathematical notation this snippet of code computes $\frac{\partial f}{\partial w_1}$. At this point you might be confused as to why we've detoured from discussing physical theories to writing simple computer programs. The connection here is that this little differentiable program can be interpreted as a physical theorem. The state space is given by tuples
\begin{align}
(x, y) \in \mathbb{R}^2,
\end{align}
and we have defined parameters, hyperparameters and an update rule \lstinline{f}. The connection is a little forced, but emphasizes the reverse bridge from programs back to physical theories: 
\begin{enumerate}
    \item States $\mathcal{S}$: Since $x_1$ and $x_2$ are both real numbers, $\mathcal{S} = \mathbb{R}^2$.
    \item Symmetries $\mathcal{G}$: There are no symmetries $\mathcal{G} = \emptyset$
    \item Operators $\mathcal{O}$: There are no operators so $\mathcal{O} = \emptyset$.
    \item Parameters $\mathcal{W}$: We have two parameters $w_1$ and $w_2$ which both take real values, so we have parameter space $\mathcal{W} = \{w_1, w_2\}$.
    \item Constraints $\mathcal{C}$: There are no constraints on the variables so $\mathcal{C} = \emptyset$.
    \item Hyperparameters $\mathcal{HP}$: We have no constraints so $\mathcal{HP} = \emptyset$
    \item Observables $\mathcal{B}$: The function $f$ defines an observable so $\mathcal{B} = \{f\}$
\end{enumerate}
In the rest of this book, we will find it useful to have a compressed table representation for the fields of a physical theory. Table~\ref{phs:theory:toy} compresses the listing above into the format we will use throughout the book.
\begin{center}
\begin{tabular}{|c|c|}
    \hline
    States & $\mathcal{S} = \mathbb{R}^2 $ \\
    \hline
    Symmetries & $\mathcal{G} = \{\}$\\
    \hline
    Operators & $\mathcal{O} = \{\} $\\
    \hline
    Parameters & $\mathcal{W} = \{\}$ \\
    \hline
    Constraints & $\mathcal{C} =  \{\}$ \\
    \hline
    Hyperparameters & $\mathcal{HP} = \{\}$  \\
    \hline
    Observables & $\mathcal{B} = \{f\}$  \\
    \hline
\end{tabular}
\label{phs:theory:toy}
\end{center}


Not all differentiable programs would make sense as physical theories, but the mapping between differentiable programs and physical theories is useful regardless since it emphasizes the computational nature of a physical theory. Through the rest of this chapter and this book, we will write down a variety of physical theories and show how to convert them to differentiable programs.



This correspondence extends further. In fact, we assert that every physical theory is expressible as a differentiable program. 
\\ \\
\textit{\textbf{The Differentiable Church-Turing Thesis:} Any physical system can be effectively modeled as a differentiable program.}
\\

We will make no attempts to prove this statement formally, but the rest of this book will attempt to prove by example by modeling a wide array of physical theories with different differentiable programs. We will note that the differentiable programs we introduce will at times have to use sophisticated operations and may be challenging to model in a standard programming language, but this is a question of engineering and programming language design.



\section{Working with Physical Theories}

Let's return to discussing physical theories at a more abstract level. What would we like to do with physical theories in practice? As a simplified model, we posit that there are a few main operations that are interesting to perform with physical theories.

\subsection{Learning the Theory from Data}

Suppose that we have a dataset of messy observations. When we learn the theory, we mean that we find a set of parameter values for the parameters in $\mathcal{W}$ that enable the physical theory to meaningfully recapitulate known facts. We write roughly

\begin{align}
\mathcal{W}^* &= \argmin_{\mathcal{W}} \mathcal{L}(\textrm{PhysicalTheory}, \mathcal{D})
\end{align}

Tools from differentiable programming such as automatic differentiation make learning parameter values much easier since we can perform gradient descent directly on a physical theory.


\subsection{Solving the Theory}

What does it mean to solve a physical theory? Very broadly speaking, a "solution" to a physical theory is characterizing the trajectory any initial set of states takes through state space. That is, we'd like to know what the state of the physical system will be at all points in time given the initial conditions of the system.

Traditionally, the solution of a physical theory meant solving the equations defining the theory analytically. This can work well for simple (and even not so simple) systems, but there's often an upper limit to the amount of complexity that can be handled analytically. This means for our system, we will have to extend this definition further.

Our definition of a solution to a physical theory is an algorithm that efficiently computes the approximate state of the system at any point in time given the initial conditions. Suppose that $\mathcal{D}$ is the collection of available data (physical observations). Let $\mathcal{L}$ be the loss, a mathematical measure of the error in fitting the theory to the data. Restated in the language we introduced above, we have a problem of learning the update operator which minimizes the loss.

\begin{align}
    \mathcal{O}^* &= \argmin_{\mathcal{O}, \mathcal{W}} \mathcal{L}(\textrm{PhysicalTheory}, \mathcal{D})
\end{align}

Differentiable physics provides a powerful collection of new tools to facilitate approximate solution of dynamic systems, including new families of techniques for high dimensional differential equation solution such as operator learning.

% %There are a few subtleties in this definition. For one, it implies that there can be more than one solution to a physical theory. This isn't as strange as it sounds at first, since even in traditional physics there's often more than one way to write the solution to a set of equations.

%One thing to note here is that solving a theory can be an extraordinarily complex task. Complex systems can a very high dimensional (or even infinite dimensional) state space. It might not be possible to derive complete solutions for many physical systems of interest, even under our expanded definition of algorithmic solutions. The classical theory of physics has extended tremendous effort in computing solutions for a variety of physical theories. 

\subsection{Taking a Measurement}

In many cases when solving a physical system isn't feasible, it is still possible to take \textit{measurements} from the physical system. It might be helpful to think of a measurement as computing a bulk property. We can for example experimentally take a measurement of average temperature of a body without solving a complete physical theory describing the motion of all atoms in the system. In many cases, similar computational measurements can be taken from physical theories. We state the problem of simulating a measurement $B$ as

\begin{align}
B^* &= B\left (\argmin_{\mathcal{W}} \mathcal{L}(\textrm{PhysicalTheory}, \mathcal{D}) \right)
\end{align}

One of the most powerful tools for simulating measurements is learned representations. Roughly stated, a representation is a function that restructures the state space of a system
\begin{align}
    \phi: \mathcal{S} \to \mathcal{S}'
\end{align}
where $\mathcal{S}'$ is called the representation space. The choice of a rich representation space can make learning measurements much easier since we can reduce the problem to finding some
\begin{align}
    B': \mathcal{S}' \to R
\end{align}
where $B'$ may have a much simpler mathematical representation than $B$. Learned representations can be viewed as a generalization of Fourier theory in which transformations $\phi$ can have more complex than the Fourier transform which allow for easy computation of certain system properties.



\section{Why Does Differentiable Physics Matter?}

Improved differentiable physics could allow for systematic design of new molecules and materials. Improved modeling of plasmas could lead to accelerated design of a functional fusion reactor. Differentiable cosmology models could help answer questions about the formation of the universe, and improved protein-ligand binding physics models could help accelerate the design of new medicine. 

\subsection{Who is the Target Audience?}

We anticipate that this book will be useful for beginning (and experienced) scientists who are interested in learning a new set of tools to understand the physical world. We provide in-depth introduction to both the basics of machine learning and the basics of physics in order to broaden the target audience. 

We anticipate that advanced undergraduates, beginning graduate students, and interested industry professionals might find this book useful. We also encourage determined younger students to give this work a try, so we've included a long appendix with needed mathematical foundations for students to tackle as they build needed skills.


\subsection{Relationship with Geometric Deep Learning}

Geometric deep learning has emerged over the last few years as a proposed organizational framework for deep learning architectures. Deep layers are viewed as transformations which preserve (perhaps approximately) a symmetry of a system. For example, shift invariance induces a convolutional networks. Time-warp invariance induces an LSTM. The framework of geometric deep learning is both more and less general than differentiable physics. Geometric deep learning can apply to problems outside the purview of physics (for example to e-commerce challenges of three dimensional inventory management), while differentiable physics can study physical systems where there is no apparent symmetry. Differentiable physics also places on emphasis on the use of techniques from the theory of programming languages, such as strong type theories. Despite these differences, both frameworks are logically close to one another and we anticipate a rich interplay between these sets of ideas.

\subsection{A Note on Mathematical Rigor}

Physicists are traditionally not rigorous with their mathematics. Often, a physicist's primary guiding intuition is their sense for how nature behaves, and not how the equations behave. This flexible approaches towards mathematical rigor has led to some amazing breakthroughs in both physics and mathematics.

However, we cannot be as intuition driven as traditional physicists. Our goal in this book is to teach techniques which can eventually be implemented in a programming language. Programming requires precision, and we can't be sloppy with what our symbols mean. At the same time, excessive formalism is avoided when it doesn't lead to improved algorithms. We'll try to walk a middle path in this chapter and in the rest of the book of using the minimal amount of rigor which is needed to be precise and clear about what our models are actually doing.

\subsection{Topics Covered}

In the rest of this book, we'll teach you how to apply the powerful tools of differentiable programming to applications in modern physics. Our discussion will be both theoretical and mathematical as well as pragmatic and practical. We will attempt to achieve a balance between these two viewpoints by interleaving dense mathematical analyses with down-to-earth and useful case studies. We hope that this blended approach will enable readers from a variety of backgrounds to approach the material in this book. Engineers and experimentalists may choose to focus on case studies in a first pass, while theorists may be more interested in the mathematical foundations we lay. We hope though that our practical readers will grow to appreciate the power of our mathematical techniques and that our theoretical readers will appreciate how useful these new tools can be in the "real" world.

Most of all, we hope that you find beauty and interest in this book. Differentiable physics offers a new paradigm of looking at much of the edifice of modern physics. We enjoyed ourselves as we wrote this book, and we hope that you share our joy while working through this book yourselves.

\end{document}





